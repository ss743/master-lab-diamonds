% !TeX spellcheck = en_GB
\section{Introduction}

Diamonds are not just known and appreciated by their beauty but also for it important technical properties like their mechanical hardness which they mainly get from their crystal structure \cite{jelezko_single_2006}. Nevertheless there are defects in their lattice structure which enable us to use diamonds more variously. Colour centres, for example, are fluorescent lattice defects which can be used for many sensing applications \cite{anleitung}. In this experiment we will examine the properties of the so-called nitrogen vacancy (NV) centre and it's sensitivity to magnetic fields.\\

Colour centres are regular defects in the lattice that absorb a particular colour in light. This not just gives a special colour to the diamond but also relates to more interesting properties like fluorescence \cite{lesik_engineering_2015}, where the lattice atoms get lifted into an excited state and then decay back into the ground state by emitting a photon with a wavelength in the visible range. The fluorescence examined here are the NV-centres that can be used to detect magnetic fields using Optical Detected Magnetic Resonance (ODMR) \cite{davis_mapping_2018}.\\

ODMR is conducted using the measurement method in which the NV-centres are excited by microwaves leading to a loss in fluorescence at a certain microwave frequency \cite{schirhagl_nitrogen-vacancy_2014}. The recorded frequency spectrum in a range around that frequency depends on the external magnetic field. Also the fluorescence spectrum of the diamond as well as its size will be measured using an optical spectrometer and a CCD camera.
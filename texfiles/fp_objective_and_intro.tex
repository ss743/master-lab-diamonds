% !TeX spellcheck = en_GB
\section{Introduction}

Diamonds are known and appreciated for their beauty and also some important technical properties like their mechanical hardness which they mainly get from their crystal structure. Nevertheless there are defects in the lattice structure which enable us to use the diamonds even more variously. Colour centres e.g. are fluorescent lattice defects which can be used for many sensing applications \cite{anleitung}. In this experiment we will examine the properties of the so-called nitrogen vacancy (NV) centre and it's sensitivity to magnetic fields.\\

Fluorescence in diamonds is created by colour centres getting lifted into an excited state and then decaying back into the ground state by emitting a photon with a wavelength in the visible range which gives the diamond their many different colours. The fluorescence of the here examined NV-centres can be used to detect magnetic fields.\\

This is done using the measurement method of optically detected magnetic resonances (ODMR) in which the NV-centre is excited by microwaves which leads to a loss in fluorescence at a certain microwave frequency. The recorded frequency spectrum in a frequency range around that frequency depends on the external magnetic field.\\

Also the fluorescence spectrum of the diamond as well as its size will be measured using an optical spectrometer and a CCD camera.
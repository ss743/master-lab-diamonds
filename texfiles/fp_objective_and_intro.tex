% !TeX spellcheck = en_GB
\section{Introduction}

Diamonds are not just known and appreciated by their beauty but also for it important technical properties like their mechanical hardness which they mainly get from their crystal structure. Nevertheless there are defects in their lattice structure which enable us to use diamonds more variously. Colour centres, for example, are fluorescent lattice defects which can be used for many sensing applications \cite{anleitung}. In this experiment we will examine the properties of the so-called nitrogen vacancy (NV) centre and it's sensitivity to magnetic fields.\\

Colour centres (CC) are  regular spacing in the lattice that absorbs a particular colour in light. Each CC involves the absence of an atom from the place it would normally occupy in the solid and the relation of an electron with such an empty place, or vacancy. Solids without colour centres may still have colour if impurity atoms or other structures that absorb light are present.

 getting lifted into an excited state and then decaying back into the ground state by emitting a photon with a wavelength in the visible range which gives the diamond their many different colours. The fluorescence of the here examined NV-centres can be used to detect magnetic fields.\\

This is done using the measurement method of optically detected magnetic resonances (ODMR) in which the NV-centre is excited by microwaves which leads to a loss in fluorescence at a certain microwave frequency. The recorded frequency spectrum in a frequency range around that frequency depends on the external magnetic field.\\

Also the fluorescence spectrum of the diamond as well as its size will be measured using an optical spectrometer and a CCD camera.
% !TeX spellcheck = en_GB
\section{Summary and Discussion}
\label{sec:summary}

%%%%IMPORTANT: Explain reasons for shifted center resonance frequency with B-field
The calibration steps were crucial in the Acquisition and conduction and data analysis of the experiment.  The knowledge of the theoretical part and the single components helped to optimize the optical signals and process the obtained data.

In therms of calibration, we notice that our microscope didn't have the same magnification as theoretically but still the $C_{f}$ founded was good enough. At the end the  determined size of the diamonds,  $24 \pm \mu m$ was in the range of the size that the manufacturer provides (20-30 microns). 

The fluorescent and its lifetimes was also in the range that the literature reported, having a really sharp peak in our emission that agrees, but an slight difference with the lifetime reported, meaning that the Calibration off the Optical spectrometer was  well realized. In our opinion with a betters source of light not scattered like the one in the laboratory, this could be done better and easily  but as show in the report the data was still well acquired and able to analyse. 

On the other hand, in the electronic components calibration it was notices as in fig \ref{fig:microstrip-trasm-eflect} that the sum of Transmission and Reflection in some point of the graph is higher than 0. Although this measurements are done in dBm and can be neglect for simplicity, we supposed that this can be corrected taking into account the transmission of the power coupler and not just its reflection as done in secction \ref{elctr}.

It was important, when triggering the SA and the Oscilloscope, to take into account  the starting time, it was not the same for both apparatus and in this triggered process some spatial resolution can be lost as happened one time, so a new set of data had to be taken again after powering off the SA, because the next measurement didn't have the same starting point "cero".

It was notice that the Faraday cage used in the APD did make a good difference in the data acquisition. In the beginning it was used as a superposed metal grid  with open endings, but after the modification we did, making it as a perfect cage with no losses endings, the noise-data range was better and made easily the analyse. 

The base of the magnets can be improve in the  Z axis, and also with some holders that keeps its position still. In the course of the measurements it was difficult to set the magnet in the Z axis holder due it  needed a lot of strength to put the magnet inside, leading to vibrations on the hole set-up and the misalignment of it.
 
 
XXXX some more ??? XXXXXXXXX

Without an external magnetic field, the expected value of $2.87GHz$ was confirmed to the  $w=2.86˘0.02qGHz$ observed in our ODMR meassuments
The split resonance of diamond.........
%may be due to the earth magnetic field, although the obtained values are on the order of three too high when taking the literature value (Bmax earth “44µT in middle Europe). Due to the fact that they are barely distinguishable, the relation holds within uncertainties. The detected contrast were consistent with the CCD-ODMR-measurements. With an external magnetic field, we were able to detect up to four magnetic resonances, corresponding to the four orientations of the NV´ center. All of them had a spatial dependency on the magnetic field. Furthermore, the calculated experienced magnetic fields Bi were always lower than the measured, external magnetic field, providing consistency. Afterwards, the theoretical, normalized orientations for each detected resonance were calculated. Due to their relative error from 25% to up to 50%, a determination of the actual orientation in the laboratory frame was not performed in this experiment. Anunexpectedpointwastheasymmetryoftheopticalcontrastinamplitude. Evenwiththe correction factor applied, positive contrasts were obtained at higher frequencies. It is unlikely thatthisisactuallyaphysicalphenomena.%%
